\documentclass{eplDoc}



\newcommand{\docType}	{Assignment 3}
\newcommand{\docDate}	{23/03/2012}
\newcommand{\docAuthor}	{gr16 : Mulders Corentin, Pelsser Francois}
\newcommand{\courseCode}{LINGI2365}
\newcommand{\courseName}{Constraint programming}
\usepackage{syntax}

\lstset{breaklines=true, breakatwhitespace=false}

\begin{document}
\maketitle
\newpage

\section{Questions}
\subsection{DC3} %1.5 pt
To express the complexity values we use the following notations from the course : $e$ is the number of constraints, $d$ is the size of the domain of the variable with the greatest domain, $r$ is the greatest arity amongst the constraints.
\subsubsection{Time complexity}
The time complexity for DC3 is $O(er^3d^{r+1})$ (if all constraints are binary then $r=2$ and the complexity becomes $O(ed^3)$). \\ 
This is because propagate can be called up to $ed$ times in propagateQueueCDC if every constraints are added to Q for every value removed from the domain of the variable with the greatest domain size. And the complexity of propagateDC3 is $O(r^2d^r)$. That is because the two forall make it so the constraint check is executed $rd$ times. And the check itself needs to try the constraint, which can be done with an $O(r)$ time complexity, on all the combinations of values for the variables and there are $d^{r-1}$ possible combinatiosn since one variable already has its value fixed.


\subsubsection{Space complexity}
The space complexity for DC3 is $O(e)$ because of the queue Q that contains the constraints for which the domain consistency is not guaranteed.  

\subsubsection{Time complexity if all constraints are domain consistent on call}

The complexity if all constraints are domain consistent at the beginning would be $O(er^3d^r)$. The complexity of the propagateDC3 function doesn't change since it still has to check the constraints to make sure they are domain consistent. However in propagateQueueCDC we will only call propagate up to $e$ times and not $ed$ since there will never be any constraint added to Q because the domains won't change, wich is why the complexity is divided by $d$. 

\subsection{valRemoveAC4} %1 pt
We test that $a \in D(x)$ because the data structure $S$ is not updated when the domains are pruned. So it is necessary to verify that $a$ is still in the domain of $x$. \\ 
If this check was removed the algorithm would still work, the $nbSupport[x,a,c]--;$ and adding $(x,a)$ to $\Delta$ wouldn't change anything since $a$ isn't in the domain of $x$ anymore already. \\ 
The time complexity of AC4 wouldn't change without this check. It would still be $O(ed^2)$. This is because the worst case scenario is unaffected by the removal of this check. The check only improves performance in some cases but not in worst case. 

\subsection{Domain and bound consistency} % 0? pt
\subsubsection{Definition of domain consistency }
The set of inconsistent values for X and Y would be defined as follow : 
$$Inc(c) = \{(X, x)|x \in D(X) \wedge \forall y \in D(Y) : \neg(x+y= a)\} \cup \{(Y, y)|y \in D(Y) \wedge \forall x \in D(X) : \neg(x+y = a)\}$$
The constraint is domain consistent with regard to D(X) and D(Y) iff $Inc(c) = \emptyset$. 
\subsubsection{Definition of bound consistency}
The constraint is bound consistent with regard to D(X) and D(Y) iff : 
$$\forall x \in D(X) : x \in [a- max(D(Y)), a- min(D(Y))]$$
 and $$\forall y \in D(Y) : y \in [a- max(D(X)), a- min(D(X))]$$
\subsubsection{Is the constraint automaticaly bound consistent if domain consistent}
Yes, the domain consistency definition of the constraint is more restrictive than the bound consistency one. If $Inc(c) = \emptyset$ for the domain consistency then the conditiosn for bound consistency are guaranteed to be met. 


\section{Problems}

\subsection{AC3 propagator : AC3Geq.co} %2.5 pt

\subsection{The channeling constraint : Channeling.co} %4.5 pt

\subsection{AC2001 propagator} %4.5 pt

\subsection{The AllDifferent constraint : AllDiffFC.co} %6 pt



\end{document}
