\documentclass{eplDoc}



\newcommand{\docType}	{Assignment 3}
\newcommand{\docDate}	{23/03/2012}
\newcommand{\docAuthor}	{gr16 : Mulders Corentin, Pelsser Francois}
\newcommand{\courseCode}{LINGI2365}
\newcommand{\courseName}{Constraint programming}
\usepackage{syntax}

\lstset{breaklines=true, breakatwhitespace=false}

\begin{document}
\maketitle
\newpage

\section{Questions}
\subsection{DC3}
To express the complexity values we use the following notations from the course : $e$ is the number of constraints, $d$ is the size of the domain of the variable with the greatest domain, $r$ is the greatest arity amongst the constraints.
\subsubsection{Time complexity}
The time complexity for DC3 is $O(er^3d^{r+1})$ (if all constraints are binary then $r=2$ and the complexity becomes $O(ed^3)$). \\ 
This is because propagate can be called up to $ed$ times in propagateQueueCDC if every constraints are added to Q for every value removed from the domain of the variable with the greatest domain size. And the complexity of propagateDC3 is $O(r^3d^r)$.
%TODO explain complexity of propagateDC3...from slide 3.34 it should be O(rd^{R+1}) and its troublesome

\subsubsection{Space complexity}
The space complexity for DC3 is $O(e)$ because of the queue Q that contains the constraints for which the domain consistency is not guaranteed.  

\subsubsection{Time complexity if all constraints are domain consistent on call}

The complexity if all constraints are domain consistent at the beginning would be $O(er^3d^r)$. The complexity of the propagateDC3 function doesn't change since it still has to check the constraints to make sure they are domain consistent. However in propagateQueueCDC we will only call propagate up to $e$ times and not $ed$ since there will never be any constraint added to Q since the domains won't change. 

\subsection{ valRemoveAC4}

\subsection{ Domain and bound consistency}

\section{Problems}

\subsection{AC3 propagator : AC3Geq.co}

\subsection{The channeling constraint : Channeling.co}

\subsection{AC2001 propagator}

\subsection{The AllDifferent constraint : AllDiffFC.co}



\end{document}
